Web browsers have become an increasingly appealing platform for application developers. Browsers make it relatively easy to deliver cross-platform applications. Web browsers have become a de facto universal operating system, and JavaScript its instruction set. Unfortunately, executing any other language than JavaScript in web browser is not usually possible. Previous approaches are either non-portable or demand extensive modifications for programs to work in the browser. Translation to JavaScript (JS) is one option but that can be challenging if the language is sufficiently different from JS. Also, debugging translated applications can be difficult.

This paper presents how languages like Scheme and Lua can be implemented in the web browser and shows how the web browsers can be extended to support multiple languages that can run in the browser simultaneously, interacting with each other seamlessly. In so doing, we hope to offer developers greater choice in languages for client-side programming.


